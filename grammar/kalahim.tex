% Options for packages loaded elsewhere
\PassOptionsToPackage{unicode}{hyperref}
\PassOptionsToPackage{hyphens}{url}
%
\documentclass[
  a4paper,
]{article}
\usepackage{amsmath,amssymb}
\usepackage{lmodern}
\usepackage{iftex}
\ifPDFTeX
  \usepackage[T1]{fontenc}
  \usepackage[utf8]{inputenc}
  \usepackage{textcomp} % provide euro and other symbols
\else % if luatex or xetex
       %%% MODIFIED: unicode-math conflics with expex; mathspects conflicts with glossaries/leipzig
  \usepackage{fontspec} % \usepackage{unicode-math}
  \defaultfontfeatures{Scale=MatchLowercase}
  \defaultfontfeatures[\rmfamily]{Ligatures=TeX,Scale=1}
  \setmainfont[]{FreeSerif}
\fi
% Use upquote if available, for straight quotes in verbatim environments
\IfFileExists{upquote.sty}{\usepackage{upquote}}{}
\IfFileExists{microtype.sty}{% use microtype if available
  \usepackage[]{microtype}
  \UseMicrotypeSet[protrusion]{basicmath} % disable protrusion for tt fonts
}{}
\usepackage{xcolor}
\IfFileExists{xurl.sty}{\usepackage{xurl}}{} % add URL line breaks if available
\IfFileExists{bookmark.sty}{\usepackage{bookmark}}{\usepackage{hyperref}}
\hypersetup{
  pdftitle={Kalahim grammar sketch},
  pdfauthor={Stefano Coretta},
  hidelinks,
  pdfcreator={LaTeX via pandoc}}
\urlstyle{same} % disable monospaced font for URLs
\usepackage[margin=1in]{geometry}
\usepackage{longtable,booktabs,array}
\usepackage{calc} % for calculating minipage widths
% Correct order of tables after \paragraph or \subparagraph
\usepackage{etoolbox}
\makeatletter
\patchcmd\longtable{\par}{\if@noskipsec\mbox{}\fi\par}{}{}
\makeatother
% Allow footnotes in longtable head/foot
\IfFileExists{footnotehyper.sty}{\usepackage{footnotehyper}}{\usepackage{footnote}}
\makesavenoteenv{longtable}
\usepackage{graphicx}
\makeatletter
\def\maxwidth{\ifdim\Gin@nat@width>\linewidth\linewidth\else\Gin@nat@width\fi}
\def\maxheight{\ifdim\Gin@nat@height>\textheight\textheight\else\Gin@nat@height\fi}
\makeatother
% Scale images if necessary, so that they will not overflow the page
% margins by default, and it is still possible to overwrite the defaults
% using explicit options in \includegraphics[width, height, ...]{}
\setkeys{Gin}{width=\maxwidth,height=\maxheight,keepaspectratio}
% Set default figure placement to htbp
\makeatletter
\def\fps@figure{htbp}
\makeatother
\setlength{\emergencystretch}{3em} % prevent overfull lines
\providecommand{\tightlist}{%
  \setlength{\itemsep}{0pt}\setlength{\parskip}{0pt}}
\setcounter{secnumdepth}{5}
\usepackage{expex}
\lingset{everyglpreamble=\bf}
\newcommand\SC[1]{\textsc{#1}}
\ifLuaTeX
  \usepackage{selnolig}  % disable illegal ligatures
\fi

\title{Kalahim grammar sketch}
\author{Stefano Coretta}
\date{}

\begin{document}
\maketitle

\hypertarget{phonology}{%
\section{Phonology}\label{phonology}}

\hypertarget{consonants}{%
\subsection{Consonants}\label{consonants}}

Kalahim has stops, fricatives, nasals, approximants, and a trill. Stops
contrast by voicing, while the fricatives are all voiceless.

\begin{longtable}[]{@{}llllll@{}}
\toprule
& labial & dental & palatal & velar & glottal \\
\midrule
\endhead
stop & p b & t d & tʃ dʒ & k g & \\
fricative & f & s & ʃ & & h \\
nasal & m & n & & & \\
approximant & w & l & j & & \\
trill & & r & & & \\
\bottomrule
\end{longtable}

\begin{enumerate}
\def\labelenumi{\arabic{enumi}.}
\tightlist
\item
  The approximant /w/ is always preceded by a consonant. For example:
  \emph{amua} /amwa/, \emph{fuanen} /fwanen/. Cf. **/wana/. Historical
  note: Single onset /w/ was lost everywhere, and the eventual resulting
  vowel combination became a diphthong or simplified to a simple vowel.
  Examples: \emph{reik-} `to turn' /reik/ \textless{} \emph{*rewik-},
  \emph{iel}- `to sing' /jel/ \textless{} /jail/ \textless{}
  \emph{*yawal-}.
\end{enumerate}

\hypertarget{vowels}{%
\subsection{Vowels}\label{vowels}}

Kalahim has a five-vowel system, with high, mid, and low heights, front,
central, and back backness. Note that the mid vowels are both mid-high.

\begin{longtable}[]{@{}llll@{}}
\toprule
& front & central & back \\
\midrule
\endhead
high & i & & u \\
mid & e & & o \\
low & & a & \\
\bottomrule
\end{longtable}

There are also the following diphthongs: /ei/, /eu/.

\hypertarget{phonotactics}{%
\subsection{Phonotactics}\label{phonotactics}}

The basic syllabic structure of Kalahim is C(C)V(C). Onsetless syllables
are only permitted in word-initial position. Independent of syllabic
affiliation, no more than two consecutive consonants are allowed in any
position in the word. Any consonant, except /tʃ/ and /dʒ/, can be
word-final.

Allowed onset clusters:

\begin{itemize}
\tightlist
\item
  pr, pl, pw, pj
\item
  tr, tl, tw, tj
\item
  kr, kl, kw, kj, ks, kʃ
\item
  br, bl, bw, bj
\item
  dr, dl, dw, dj
\item
  gr, gl, gw, gj
\item
  sp, st, sk, sw, sj
\item
  ʃp, ʃt, ʃk, ʃl, ʃw, ʃj
\item
  fl, fr, fw, fj
\item
  hw, hj
\item
  mw, mj
\item
  nw, nj
\item
  rw, rj
\item
  lw, lj
\end{itemize}

\hypertarget{stress}{%
\subsection{Stress}\label{stress}}

Stress is intensive and it falls always on the penultimate syllable of
the word in plurisyllabic words, and in the only syllable of
monosyllabic words.

\hypertarget{writing-system}{%
\section{Writing system}\label{writing-system}}

The transliteration is as follows.

Consonants:

\begin{longtable}[]{@{}llllll@{}}
\toprule
& labial & dental & palatal & velar & glottal \\
\midrule
\endhead
stop & p b & t d & c j & k g & \\
fricative & f & s & sh & & h \\
nasal & m & n & & & \\
approximant & u & l & i & & \\
trill & & r & & & \\
\bottomrule
\end{longtable}

Vowels:

\begin{longtable}[]{@{}llll@{}}
\toprule
& front & central & back \\
\midrule
\endhead
high & i/y & & u \\
mid & e/ai & & o/au \\
low & & a & \\
\bottomrule
\end{longtable}

\begin{itemize}
\tightlist
\item
  {[}i, e, o{]} are written as ⟨i, e, o⟩ and ⟨y, ai, au⟩ for historical
  reasons.

  \begin{itemize}
  \tightlist
  \item
    Alternative transliterations for ⟨ai, au⟩ are ⟨ê, ô⟩.
  \end{itemize}
\end{itemize}

\hypertarget{morphosyntax}{%
\section{Morphosyntax}\label{morphosyntax}}

\hypertarget{word-order}{%
\subsection{Word order}\label{word-order}}

The general word order is SVO. Complements follow the noun they refer
to, but adjectives precede it.

\ex \begingl \glpreamble Blin manai acaroni.// \gla blin mainai
acaro-ni// \glb strong warrior east-GEN:SG// \glft `The strong warrior
of the East.'// \endgl \xe

\hypertarget{verbs}{%
\subsection{Verbs}\label{verbs}}

\emph{Revision}: Verbs end in vowel (are there C-final verbs? what about
nouns?). In the Present, the vowel coalesces with the Present Tense
suffix \emph{-u}: a-u \textgreater{} o, e-u \textgreater{} ie, i-u
\textgreater{} y, o-u \textgreater{} o, u-u \textgreater{} u. Past and
Future are analytic? Past uses \emph{lusho} `to go' as auxiliary (new
`to go' is different). Future is just the verb stem plus the Subj
suffixes.

Infinitive: -sh (\textless{} -tʲ). Hypothetical from -bʷi, a form of
future?

Passive: it was -s, now it has Non-Past form (Present and Future) which
is original Gnomic (that takes on future meaning in Active). Passive
Past is with auxiliary \emph{dlanti} `to come'.

\emph{reika} `to turn'

\begin{longtable}[]{@{}lllll@{}}
\toprule
Active & present & past & future & hypothetical \\
\midrule
\endhead
1s & reiko-go & lurgo reika-s & reika-go & lurby-go reika-s \\
2s & reiko-ni & lurni reika-s & reika-ni & lurby-ni reika-s \\
3sm & reiko-lo & luelo reika-s & reika-lo & lurby-lo reika-s \\
3sf & reiko-re & luere reika-s & reika-re & lurby-re reika-s \\
3sn & reiko-lah & luelah reika-s & reika-lah & lurby-lah reika-s \\
1pl & reiko-min & lurmin reika-s & reika-min & lurby-min reika-s \\
2pl & reiko-ty & lusty reika-s & reika-ty & lurby-ty reika-s \\
3plm & reiko-su & luesu reika-s & reika-su & lurby-su reika-s \\
3plf & reiko-shi & lueshi reika-s & reika-shi & lurby-shi reika-s \\
3pln & reiko-sah & luesah reika-s & reika-sah & lurby-sah reika-s \\
\bottomrule
\end{longtable}

\begin{longtable}[]{@{}lll@{}}
\toprule
Passive & non-past & past \\
\midrule
\endhead
1s & reika-gos & lango reika-s \\
2s & reika-nis & leni reika-s \\
3sm & reika-los & lelo reika-s \\
3sf & reika-res & lere reika-s \\
3sn & reika-les & lelah reika-s \\
1pl & reika-meis & lemin reika-s \\
2pl & reika-tys & lanty reika-s \\
3plm & reika-sus & lesu reika-s \\
3plf & reika-shis & leshi reika-s \\
3pln & reika-ses & lesah reika-s \\
\bottomrule
\end{longtable}

\ex \begingl \glpreamble Lango (done)ganis nide.// \gla lan-go
(done)gani-s ni-de// \glb PST:PASS-1sg save-PASS.PRT 2s-ABL:SG//
\glft `I was saved by you.'// \endgl \xe

Present participles are formed with the suffix -\emph{m} which is
attached to the verb root and they can be declined according the to
consonant declination.

\ex \begingl \glpreamble Catieni shonamen// \gla catie-ni shona-m-en//
\glb person-GEN:SG think-PTCP-GEN:SG// \glft `Of the thinking person.'//
\endgl \xe

\hypertarget{copula}{%
\subsubsection{Copula}\label{copula}}

There are two copulas in Kalahim, \emph{om-} (\textless{} \emph{aumi-})
and \emph{ash-} (\textless{} \emph{ashi-}). The first is used with the
Predicative case and refers to a permanent or substantial feature of the
subject. For example, \emph{omy ned-es} `I am (a) man'. \emph{ash-} is
used with temporary states and feelings, like in \emph{ashe cul} `it is
sweet'.

\begin{longtable}[]{@{}lllll@{}}
\toprule
aum- & present & past & future & hypothetical \\
\midrule
\endhead
1s & omy & on-go/ue-go & om-ei-go & om-by-go \\
2s & omy-ni & ue-ni & om-ei-ni & om-by-ni \\
3sm & omy & ue-lo & om-ei-lo & om-by-lo \\
3sf & omy & ue-re & om-ei-re & om-by-re \\
3sn & omy & ue-lah & om-ei-lah & om-by-lah \\
1pl & omy-min & ue-min & om-ei-min & om-by-min \\
2pl & omy-ty & on-ty/ue-ty & om-ei-ty & om-by-ty \\
3plm & omy-su & ue-su & om-ei-su & om-by-su \\
3plf & omy-shi & ue-shi & om-ei-shi & om-by-shi \\
3pln & omy-sah & ue-sah & om-ei-sah & om-by-sah \\
\bottomrule
\end{longtable}

Present should have stem in -ie but it is reduced to -e because of
preceding -sh (only when unstressed).

\begin{longtable}[]{@{}lllll@{}}
\toprule
ash- & present & past & future & hypothetical \\
\midrule
\endhead
1s & ashe & e-go & ash-o-go & e-by-go \\
2s & ashie-ni & e-ni & ash-o-ni & e-by-ni \\
3sm & ashe & ash-lo & ash-o-lo & e-by-lo \\
3sf & ashe & e-re & ash-o-re & e-by-re \\
3sn & ashe & ash-lah & ash-o-lah & e-by-lah \\
1pl & ashie-min & e-min & ash-o-min & e-by-min \\
2pl & ashie-ty & ash-ty & ash-o-ty & e-by-ty \\
3plm & ashie-su & e-shu & ash-o-su & e-by-su \\
3plf & ashie-shi & e-shi & ash-o-shi & e-by-shi \\
3pln & ashie-sah & e-shah & ash-o-sah & e-by-sah \\
\bottomrule
\end{longtable}

\hypertarget{locative-om-}{%
\subsubsection{\texorpdfstring{Locative
\emph{om-}}{Locative om-}}\label{locative-om-}}

The verb \emph{om-}, which means `to be in some place' derives from
\emph{aum-}, although its paradigm has regularised and follows the
default verbal conjugation. In the 3rd Neuter Singular and Plural it is
also used together with the Predicative case as an expletive to mean
`There is/are'.

\ex \begingl \glpreamble Omogo Limlin'ija.// \gla omo-go Limlin'-ija//
\glb be.LOC.PRS-1s Limlin-LOC:SG// \glft `I am in Limlin.'// \endgl \xe

\ex \begingl \glpreamble Luesah omis jekines.// \gla luesah omi-s
jeki-nes// \glb PST.3pl.N be.EXPL-PST.PRTC monkey-PRED:PL// \glft `There
were (some) monkeys.'// \endgl \xe

\hypertarget{negation}{%
\subsubsection{Negation}\label{negation}}

A verb is negated by placing the negative adverb \emph{an} after the
verb.

\ex \begingl \glpreamble Dlantigo an fladriba.// \gla dlanti-go an
fladri-ba// \glb come.FUT-1s NEG forest-ALL:SG// \glft `I will not come
to the forest.'// \endgl \xe

\hypertarget{nouns}{%
\subsection{Nouns}\label{nouns}}

Nouns end either in a vowel or a consonant and follow their respective
class declension. Archaic forms of suffixes are given in square
brackets. The predicative plural ending has two forms, -\emph{nes} and
-\emph{nesse} which tend to be respectively the standard in the spoken
and written language.

The predicative is used with the verb \emph{aum-} `to be', which has an
existential meaning (while \emph{ash}- is more generally stative).

\begin{longtable}[]{@{}lll@{}}
\toprule
vowel & singular & plural \\
\midrule
\endhead
nominative & {[}varies{]} & -m \\
accusative & -l {[}arc. -lu{]} & -le {[}arc.-leo{]} \\
predicative & -sse & -nes(se) \\
dative & -vo & -vi \\
instrumental & -ru {[}arc.-kru{]} & -du {[}arc.-dru{]} \\
genitive & -ni {[}arc.-nig{]} & -nek \\
locative & -ja {[}arc.-jai{]} & -ce \\
allative & -ba & -pe \\
ablative & -da {[}arc.-de{]} & -te \\
vocative & -ku & -ho \\
\bottomrule
\end{longtable}

\begin{longtable}[]{@{}lll@{}}
\toprule
consonant & singular & plural \\
\midrule
\endhead
nominative & {[}varies{]} & -am \\
accusative & -olu & -elo \\
predicative & -es {[}arc.-esse{]} & -enes \\
dative & -ol & -ive \\
instrumental & -ur {[}arc.-ruk{]} & -udu {[}arc.-udru{]} \\
genitive & -en & -ek \\
locative & -ija & -ice \\
allative & -iba & -ipe \\
ablative & -ida {[}arc.-eda{]} & -ete \\
vocative & -uk & -oho \\
\bottomrule
\end{longtable}

\hypertarget{subordination}{%
\subsection{Subordination}\label{subordination}}

\textbf{Complement clauses} are introduced by the particle \emph{nai}.
All complement clauses are finite (they always have a finite verb).

\emph{Kabasiesu nai korashiego.}

\begin{verbatim}
kabas-ie-su   nai korash-ie-go
know-PST-3plm SUB be.happy-PST-1s
'They knew I was/have been happy.'
\end{verbatim}

\emph{Tokago nai noirgah-a-go nil.}

\begin{verbatim}
tok-a-go    nai sibu-a-go   ni-l
want-PRS-1s SUB meet-PRS-1s 2s-ACC:SG
'I want to meet you.'
\end{verbatim}

Subject complement clauses are preceded by an independent clause of the
form `be.PRED + ADJ/PTCP'.

\emph{Omalah pushaitam nai sibuemin.}

\begin{verbatim}
om-a-lah        pushait-am  nai sibu-o-min
be.PRED-PRS-3sn scare-PTCP  SUB meet-PST-1pl
'It scares me that we will meet.'
\end{verbatim}

\textbf{Adverbial clauses} are introduced by a variety of adverbial
subordinating conjunctions.

\textbf{Relative clauses} are introduced by \emph{ki} and a resumptive
pronoun is used. The resumptive pronoun in the relative clause is in the
case and position it would be if the clause were independent. A
nominative resumptive pronoun is generally omitted. If the antecedent is
the object of the main clause and the subject of the relative close,
\emph{ki} and the subject pronoun are omitted and the antecedent is in
the nominative case.

\emph{Lonum ki voinalo korashalo.}

\begin{verbatim}
lonu-m  ki  voin-a-lo     korash-a-lo
boy-PL  REL play-PRS-3sm  be.happy-PRS-3sm
'The boy who plays is happy.'
\end{verbatim}

\emph{Nem korashielo mol noirgahielo lonu voinielo.}

\begin{verbatim}
nem       korash-ie-lo    mol   noirgah-ie-lo lonu    voin-ie-lo
yesterday be.hapy-PST-3sm GRND  see-PST-3sm   boy.NOM play-PST-3sm
'Yesterday he rejoiced in seeing the boy who was playing.'
\end{verbatim}

Cf. with the object complement clause \emph{Nem korashielo mol
noirgahielo nai lonu voinielo} `Yesterday he rejoiced in seeing that the
boy was playing.'

\emph{Firkiniego catieru ki noirgahieni laholu.}

\begin{verbatim}
firkin-ie-go  catie-ru        ki  noirgah-ie-ni lah-olu
speak-PST-1s  person-INST:SG  REL see-PST-2s    3sn-ACC:SG
'I spoke with the person you saw.'
\end{verbatim}

\emph{Noirgahiego catiel ki firkinieni lahur.}

\begin{verbatim}
noirgah-ie-go catie-l       ki  firkin-ie-ni  lah-ur
see-PST-1s    person-ACC:SG REL speak-PST-2s  3sn-INST:SG
'I saw the person you spoke to.'
\end{verbatim}

\hypertarget{glossed-examples}{%
\section{Glossed examples}\label{glossed-examples}}

\ex \begingl \glpreamble Ashani lof namshal gonji shimalah ki mallabalah
vinedolu, e Firniluk goniguk!// \gla ash-a-ni lof namsha-l gonji
shim-a-lah ki mallab-a-lah vined-olu e Firnil-uk gonig-uk//
\glb be-PRS-2s like star-ACC:SG so.much shine-PRS-3sn CONJ
obscure-PRS-3sn other-ACC:PL oh Firnil-VOC:SG my-VOC:SG// \glft `You are
like a star that is so bright to obscure the others, oh my Firnil!'//
\endgl \xe

\end{document}
